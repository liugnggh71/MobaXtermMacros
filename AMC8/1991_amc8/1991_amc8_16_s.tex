

The $16$ squares on a piece of paper are numbered as shown in the diagram. While lying on a table, the paper is folded in half four times in the following sequence:

(1) fold the top half over the bottom half

(2) fold the bottom half over the top half

(3) fold the right half over the left half

(4) fold the left half over the right half.

Which numbered square is on top after step $4$?

\begin{figure}[H]
\centering
\begin{asy}
import "./olympiad.asy" as olympiad;
 unitsize(18); for(int a=0; a<5; ++a)  {   draw((a,0)--(a,4));  } for(int b=0; b<5; ++b)  {   draw((0,b)--(4,b));  } label("$1$",(0.5,3.1),N); label("$2$",(1.5,3.1),N); label("$3$",(2.5,3.1),N); label("$4$",(3.5,3.1),N); label("$5$",(0.5,2.1),N); label("$6$",(1.5,2.1),N); label("$7$",(2.5,2.1),N); label("$8$",(3.5,2.1),N); label("$9$",(0.5,1.1),N); label("$10$",(1.5,1.1),N); label("$11$",(2.5,1.1),N); label("$12$",(3.5,1.1),N); label("$13$",(0.5,0.1),N); label("$14$",(1.5,0.1),N); label("$15$",(2.5,0.1),N); label("$16$",(3.5,0.1),N); 
\end{asy}
\end{figure}

$\text{(A)}\ 1 \qquad \text{(B)}\ 9 \qquad \text{(C)}\ 10 \qquad \text{(D)}\ 14 \qquad \text{(E)}\ 16$
\\
Solution
\\
Suppose we undo each of the four folds, considering just the top square until we completely unfold the paper.  $x$ will be marked in the square if the face that shows after all the folds is face up, $y$ if that face is facing down.

Step 0: 
\begin{figure}[H]
\centering
\begin{asy}
import "./olympiad.asy" as olympiad;
 unitsize(18); draw((0,0)--(1,0)--(1,1)--(0,1)--cycle); label("$x$",(0.5,0.1),N); 
\end{asy}
\end{figure}
 
Step 1: 
\begin{figure}[H]
\centering
\begin{asy}
import "./olympiad.asy" as olympiad;
 unitsize(18); draw((0,0)--(2,0)--(2,1)--(0,1)--cycle); draw((1,0)--(1,1)); label("$y$",(0.5,0.1),N); 
\end{asy}
\end{figure}
Step 2: 
\begin{figure}[H]
\centering
\begin{asy}
import "./olympiad.asy" as olympiad;
unitsize(18); draw((0,0)--(4,0)--(4,1)--(0,1)--cycle);  draw((1,0)--(1,1)); draw((2,0)--(2,1)); draw((3,0)--(3,1)); label("$y$",(0.5,0.1),N); 
\end{asy}
\end{figure}
Step 3: 
\begin{figure}[H]
\centering
\begin{asy}
import "./olympiad.asy" as olympiad;
unitsize(18); draw((0,0)--(4,0)--(4,2)--(0,2)--cycle); draw((0,1)--(4,1)); draw((1,0)--(1,2)); draw((2,0)--(2,2)); draw((3,0)--(3,2)); label("$y$",(0.5,1.1),N); 
\end{asy}
\end{figure}
Step 4: 
\begin{figure}[H]
\centering
\begin{asy}
import "./olympiad.asy" as olympiad;
unitsize(18); for(int a=0; a<5; ++a)  {   draw((a,0)--(a,4));  } for(int b=0; b<5; ++b)  {   draw((0,b)--(4,b));  } label("$y$",(0.5,1.1),N); 
\end{asy}
\end{figure}
The marked square is in the same spot as the number $9\rightarrow \boxed{\text{B}}$.
