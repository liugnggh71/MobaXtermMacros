

A "domino" is made up of two small squares: 
\begin{figure}[H]
\centering
\begin{asy}
import "./olympiad.asy" as olympiad;
 unitsize(12); fill((0,0)--(1,0)--(1,1)--(0,1)--cycle,black);  draw((1,1)--(2,1)--(2,0)--(1,0)); 
\end{asy}
\end{figure}
Which of the "checkerboards" illustrated below CANNOT be covered exactly and completely by a whole number of non-overlapping dominoes?

\begin{figure}[H]
\centering
\begin{asy}
import "./olympiad.asy" as olympiad;
unitsize(12); fill((0,0)--(1,0)--(1,1)--(0,1)--cycle,black); fill((1,1)--(1,2)--(2,2)--(2,1)--cycle,black); fill((2,0)--(3,0)--(3,1)--(2,1)--cycle,black); fill((3,1)--(4,1)--(4,2)--(3,2)--cycle,black); fill((0,2)--(1,2)--(1,3)--(0,3)--cycle,black); fill((2,2)--(2,3)--(3,3)--(3,2)--cycle,black); draw((0,0)--(0,3)--(4,3)--(4,0)--cycle); draw((6,0)--(11,0)--(11,3)--(6,3)--cycle); fill((6,0)--(7,0)--(7,1)--(6,1)--cycle,black); fill((8,0)--(9,0)--(9,1)--(8,1)--cycle,black); fill((10,0)--(11,0)--(11,1)--(10,1)--cycle,black); fill((7,1)--(7,2)--(8,2)--(8,1)--cycle,black); fill((9,1)--(9,2)--(10,2)--(10,1)--cycle,black); fill((6,2)--(6,3)--(7,3)--(7,2)--cycle,black); fill((8,2)--(8,3)--(9,3)--(9,2)--cycle,black); fill((10,2)--(10,3)--(11,3)--(11,2)--cycle,black); draw((13,-1)--(13,3)--(17,3)--(17,-1)--cycle); fill((13,3)--(14,3)--(14,2)--(13,2)--cycle,black); fill((15,3)--(16,3)--(16,2)--(15,2)--cycle,black); fill((14,2)--(15,2)--(15,1)--(14,1)--cycle,black); fill((16,2)--(17,2)--(17,1)--(16,1)--cycle,black); fill((13,1)--(14,1)--(14,0)--(13,0)--cycle,black); fill((15,1)--(16,1)--(16,0)--(15,0)--cycle,black); fill((14,0)--(15,0)--(15,-1)--(14,-1)--cycle,black); fill((16,0)--(17,0)--(17,-1)--(16,-1)--cycle,black); draw((19,3)--(24,3)--(24,-1)--(19,-1)--cycle,black); fill((19,3)--(20,3)--(20,2)--(19,2)--cycle,black); fill((21,3)--(22,3)--(22,2)--(21,2)--cycle,black); fill((23,3)--(24,3)--(24,2)--(23,2)--cycle,black); fill((20,2)--(21,2)--(21,1)--(20,1)--cycle,black); fill((22,2)--(23,2)--(23,1)--(22,1)--cycle,black); fill((19,1)--(20,1)--(20,0)--(19,0)--cycle,black); fill((21,1)--(22,1)--(22,0)--(21,0)--cycle,black); fill((23,1)--(24,1)--(24,0)--(23,0)--cycle,black); fill((20,0)--(21,0)--(21,-1)--(20,-1)--cycle,black); fill((22,0)--(23,0)--(23,-1)--(22,-1)--cycle,black); draw((26,3)--(29,3)--(29,-3)--(26,-3)--cycle); fill((26,3)--(27,3)--(27,2)--(26,2)--cycle,black); fill((28,3)--(29,3)--(29,2)--(28,2)--cycle,black); fill((27,2)--(28,2)--(28,1)--(27,1)--cycle,black); fill((26,1)--(27,1)--(27,0)--(26,0)--cycle,black); fill((28,1)--(29,1)--(29,0)--(28,0)--cycle,black); fill((27,0)--(28,0)--(28,-1)--(27,-1)--cycle,black); fill((26,-1)--(27,-1)--(27,-2)--(26,-2)--cycle,black); fill((28,-1)--(29,-1)--(29,-2)--(28,-2)--cycle,black); fill((27,-2)--(28,-2)--(28,-3)--(27,-3)--cycle,black); 
\end{asy}
\end{figure}

$\text{(A)}\ 3\times 4 \qquad \text{(B)}\ 3\times 5 \qquad \text{(C)}\ 4\times 4 \qquad \text{(D)}\ 4\times 5 \qquad \text{(E)}\ 6\times 3$
