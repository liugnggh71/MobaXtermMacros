

In a magic triangle, each of the six whole numbers $10-15$ is placed in one of the circles so that the sum, $S$, of the three numbers on each side of the triangle is the same. The largest possible value for $S$ is

\begin{figure}[H]    
\centering         
\begin{asy}         
import "./olympiad.asy" as olympiad;
size(0,120);         
         
draw(circle((0,0),1)); draw(dir(60)--6*dir(60)); draw(circle(7*dir(60),1)); draw(8*dir(60)--13*dir(60)); draw(circle(14*dir(60),1)); draw((1,0)--(6,0)); draw(circle((7,0),1)); draw((8,0)--(13,0)); draw(circle((14,0),1)); draw(circle((10.5,6.0621778264910705273460621952706),1)); draw((13.5,0.86602540378443864676372317075294)--(11,5.1961524227066318805823390245176)); draw((10,6.9282032302755091741097853660235)--(7.5,11.258330249197702407928401219788)); 
\end{asy}         
\end{figure}  
$\text{(A)}\ 36 \qquad \text{(B)}\ 37 \qquad \text{(C)}\ 38 \qquad \text{(D)}\ 39 \qquad \text{(E)}\ 40$
\\
Solution
\\
Let the number in the top circle be $a$ and then $b$, $c$, $d$, $e$, and $f$, going in clockwise order. Then, we have \[S=a+b+c\]\[S=c+d+e\]\[S=e+f+a\]
Adding these equations together, we get

\begin{align*} 3S &= (a+b+c+d+e+f)+(a+c+e) \\ &= 75+(a+c+e) \\ \end{align*}
where the last step comes from the fact that since $a$, $b$, $c$, $d$, $e$, and $f$ are the numbers $10-15$ in some order, their sum is $10+11+12+13+14+15=75$

The left hand side is divisible by $3$ and $75$ is divisible by $3$, so $a+c+e$ must be divisible by $3$. The largest possible value of $a+c+e$ is then $15+14+13=42$, and the corresponding value of $S$ is $\frac{75+42}{3}=39$, which is choice $\boxed{\text{D}}$.

It turns out this sum is attainable if you let \[a=15\]\[b=10\]\[c=14\]\[d=12\]\[e=13\]\[f=11\]
