

\begin{figure}[H]    
\centering         
\begin{asy}         
import "./olympiad.asy" as olympiad;
size(0,120);         

unitsize(13); draw((0,0)--(20,0)); draw((0,0)--(0,15)); draw((0,3)--(-1,3)); draw((0,6)--(-1,6)); draw((0,9)--(-1,9)); draw((0,12)--(-1,12)); draw((0,15)--(-1,15)); fill((2,0)--(2,15)--(3,15)--(3,0)--cycle,black); fill((4,0)--(4,12)--(5,12)--(5,0)--cycle,black); fill((6,0)--(6,9)--(7,9)--(7,0)--cycle,black); fill((8,0)--(8,9)--(9,9)--(9,0)--cycle,black); fill((10,0)--(10,15)--(11,15)--(11,0)--cycle,black); label("A",(2.5,-.5),S); label("B",(4.5,-.5),S); label("C",(6.5,-.5),S); label("D",(8.5,-.5),S); label("F",(10.5,-.5),S); label("Grade",(15,-.5),S); label("$1$",(-1,3),W); label("$2$",(-1,6),W); label("$3$",(-1,9),W); label("$4$",(-1,12),W); label("$5$",(-1,15),W); 
\end{asy}         
\end{figure}         

The bar graph shows the grades in a mathematics class for the last grading period. If A, B, C, and D are satisfactory grades, what fraction of the grades shown in the graph are satisfactory?

$\text{(A)}\ \frac{1}{2} \qquad \text{(B)}\ \frac{2}{3} \qquad \text{(C)}\ \frac{3}{4} \qquad \text{(D)}\ \frac{4}{5} \qquad \text{(E)}\ \frac{9}{10}$
\\
Solution
\\
To get the fraction, we need to find the number of people who got grades that are "satisfactory" over the total number of people.

Finding the number of people who got acceptable grades is pretty easy. 5 people got A's, 4 people got B's, 3 people got C's and 3 people got D's. Adding this up, we just have $5+4+3+3 = 15$.

So we know the top of the fraction is 15. Only 5 people got "unacceptable" scores, so there are $15 + 5 = 20$ scores.

$\frac{15}{20}=\frac{3}{4}$ is our fraction, so $\boxed{\text{C}}$ is the answer.
