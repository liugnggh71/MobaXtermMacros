

The difference between a $6.5\%$ sales tax and a $6\%$ sales tax on an item priced at $\$20$ before tax is

$\text{(A)}$ $\$.01$

$\text{(B)}$ $\$.10$

$\text{(C)}$ $\$ .50$

$\text{(D)}$ $\$ 1$

$\text{(E)}$ $\$10$
\\
Solution
\\
The most straightforward method would be to calculate both prices, and subtract. But there's a better method...

Before we start, it's always good to convert the word problems into expressions, we can solve.

So we know that the price of the object after a $6.5\%$ increase will be $20 \times 6.5\%$, and the price of it after a $6\%$ increase will be $20 \times 6\%$. And what we're trying to find is $6.5\% \times 20 - 6\% \times 20$, and if you have at least a little experience in the field of algebra, you'll notice that both of the items have a common factor, $20$, and we can factor the expression into \begin{align*} (6.5\% - 6\% ) \times 20 &= (.5\% )\times 20 \\ &= \frac{.5}{100}\times 20 \\ &= \frac{1}{200}\times 20 \\ &= .10 \\ \end{align*}
$.10$ is choice $\boxed{\text{B}}$
