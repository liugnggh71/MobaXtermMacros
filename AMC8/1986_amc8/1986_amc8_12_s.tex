

The table displays the grade distribution of the $ 30$ students in a mathematics class on the last two tests. For example, exactly one student received a "D" on Test 1 and a "C" on Test 2. What percent of the students received the same grade on both tests?

\begin{figure}[H]
\centering
\begin{asy}
size(200);
draw((0,0)--(5,0));
draw((0,1)--(5,1));
draw((0,2)--(5,2));
draw((0,3)--(5,3));
draw((0,4)--(5,4));
draw((0,5)--(5,5));
draw((0,0)--(0,5));
draw((1,0)--(1,5));
draw((2,0)--(2,5));
draw((3,0)--(3,5));
draw((4,0)--(4,5));
draw((5,0)--(5,5));
draw((0,5)--(-2,7));
label("F",(0,0.5),W);
label("D",(0,1.5),W);
label("C",(0,2.5),W);
label("B",(0,3.5),W);
label("A",(0,4.5),W);
label("A",(0.5,5),N);
label("B",(1.5,5),N);
label("C",(2.5,5),N);
label("D",(3.5,5),N);
label("F",(4.5,5),N);
label("0",(0.5,0),N);
label("0",(0.5,1),N);
label("1",(0.5,2),N);
label("1",(0.5,3),N);
label("2",(0.5,4),N);
label("0",(1.5,0),N);
label("0",(1.5,1),N);
label("3",(1.5,2),N);
label("4",(1.5,3),N);
label("2",(1.5,4),N);
label("2",(2.5,0),N);
label("1",(2.5,1),N);
label("5",(2.5,2),N);
label("3",(2.5,3),N);
label("1",(2.5,4),N);
label("1",(3.5,0),N);
label("1",(3.5,1),N);
label("2",(3.5,2),N);
label("0",(3.5,3),N);
label("0",(3.5,4),N);
label("0",(4.5,0),N);
label("1",(4.5,1),N);
label("0",(4.5,2),N);
label("0",(4.5,3),N);
label("0",(4.5,4),N);
label("TEST 2",(1,6),N);
label("TEST 1",(-2,5),SW);
\end{asy}
\end{figure}


\[ \textbf{(A)}\ 12 \% \qquad
\textbf{(B)}\ 25 \% \qquad
\textbf{(C)}\ 33 \frac{1}{3} \% \qquad
\textbf{(D)}\ 40 \% \qquad
\textbf{(E)}\ 50 \% \qquad
\]
\\
Solution
\\
We need to find the number of those who did get the same on both tests over 30 (the number of students in the class).

So, we have \[\frac{2 + 4 + 5 + 1}{30}\]
Which simplifies to \[\frac{12}{30} = \frac{4}{10} = \frac{40}{100} = 40 \%\]
$\boxed{\text{D}}$