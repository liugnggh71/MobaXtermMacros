

The large circle has diameter $ \overline{AC}$. The two small circles have their centers on $ \overline{AC}$ and just touch at $ O$, the center of the large circle. If each small circle has radius $ 1$, what is the value of the ratio of the area of the shaded region to the area of one of the small circles?

\begin{figure}[H]
\centering
\begin{asy}
import "olympiad.asy" as olympiad;
size(200);
pair A=(-2,0), O=origin, C=(2,0);
path X=Arc(O,2,0,180), Y=Arc((-1,0),1,180,0), Z=Arc((1,0),1,180,0), N=X..Y..Z..cycle;
filldraw(N, black, black);
draw(reflect(A,C)*N);
draw(A--C, dashed);

label("A",A,W);
label("C",C,E);
label("O",O,SE);
dot((-1,0));
dot(O);
dot((1,0));
label("1",(-1,0),NE);
label("1",(1,0),NW);
\end{asy}
\end{figure}

\[ \textbf{(A)}\ \text{between }\frac{1}{2} \text{ and }1 \qquad
\textbf{(B)}\ 1 \qquad
\textbf{(C)}\ \text{between 1 and }\frac{3}{2} \qquad
\textbf{(D)}\ \text{between }\frac{3}{2} \text{ and }2 \\
\textbf{(E)}\ \text{cannot be determined from the information given}
\]
\\
Solution
\\
The small circle has radius $1$, thus its area is $\pi$.

The large circle has radius $2$, thus its area is $4\pi$.

The area of the semicircle above $AC$ is then $2\pi$.

The part that is not shaded are two small semicircles. Together, these form one small circle, hence their total area is $\pi$. This means that the area of the shaded part is $2\pi-\pi=\pi$. This is equal to the area of a small circle, hence the correct answer is $\boxed{\text{(B)}\ 1}$.
