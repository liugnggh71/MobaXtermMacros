

Given that all angles shown are marked, the perimeter of the polygon shown is

\begin{figure}[H]
\centering
\begin{asy}
size(200);
draw((0,0)--(0,6)--(8,6)--(8,3)--(4,3)--(4,0)--cycle);
label("6",(0,3),W);
label("8",(4,6),N);
\end{asy}
\end{figure}

\[ \textbf{(A)}\ 14 \qquad
\textbf{(B)}\ 20 \qquad
\textbf{(C)}\ 28 \qquad
\textbf{(D)}\ 48 \qquad
\textbf{(E)}\ \text{cannot be determined from the information given} \qquad
\]
\\
Solution 1
\\
For the segments parallel to the side with side length 8, let's call those two segments $a$ and $b$, the longer segment being $b$, the shorter one being $a$.

For the segments parallel to the side with side length 6, let's call those two segments $c$ and $d$, the longer segment being $d$, the shorter one being $c$.

So the perimeter of the polygon would be...

$8 + 6 + a + b + c + d$

Note that $a + b = 8$, and $c + d = 6$.

Now we plug those in: \begin{align*} 8 + 6 + a + b + c + d &= 8 + 6 + 8 + 6 \\ &= 14 \times 2 \\ &= 28 \\ \end{align*}
28 is $\boxed{\text{C}}$.
\\
Solution 2
\\
\begin{figure}[H]
\centering
\begin{asy}
import "olympiad.asy" as olympiad;
unitsize(12); draw((0,0)--(0,6)--(8,6)--(8,3)--(2.7,3)--(2.7,0)--cycle); label("$6$",(0,3),W); label("$8$",(4,6),N); draw((8,3)--(8,0)--(2.7,0),dashed); 
\end{asy}
\end{figure}
The perimeter of the requested region is the same as the perimeter of the rectangle with the dashed portion. This makes the answer $2(6+8)=28\rightarrow \boxed{\text{C}}$
